% Preamble {{{1
% See <https://tex.stackexchange.com/q/29181>.  There's some issue that causes captions
% like "Listing 2.: Foo".  setting the `numbers` option like this fixes it.
\documentclass[a4paper,numbers=noenddot]{scrartcl}

\usepackage[utf8]{inputenc} % Assume this file is encoded in UTF-8.
\usepackage[T1]{fontenc}    % Don't fake umlauts etc.
\usepackage{lmodern}        % Use the lmodern font (http://tex.stackexchange.com/a/65103).
\usepackage{microtype}      % Better microtypography (http://www.ctan.org/pkg/microtype)
\usepackage{hyperref}       % Clickable hyperlinks, load before `glossaries`
\usepackage{cleveref}       % Use `\cref{fig:foo}` instead of `figure~\ref{fig:foo}`.
                            % Load after `hyperref`.  See
                            % <https://tex.stackexchange.com/q/36295>
\usepackage{comment}        % Comment out sections of text.
\usepackage{mathtools}      % Improved facilities for typesetting mathematical formulae
\usepackage{dirtytalk}      % ...

% Prevent floats from moving into another section (https://www.ctan.org/pkg/placeins).
\usepackage[above, below, section]{placeins}

% Provides the `\captionof` command for typesetting captions outside of floats.  Includes
% the [newfloat package](https://www.ctan.org/pkg/newfloat), which minted is set up to
% use.
\usepackage{caption} % See <https://www.ctan.org/pkg/caption>.
% \captionsetup{labelsep=colon}

% See <https://tex.stackexchange.com/q/10684/78512>.  TODO.
\usepackage{enumitem}
\setlist{noitemsep}

% TODO.  See <https://tex.stackexchange.com/q/119513>.
\crefname{appsec}{Appendix}{Appendices}

% Use ISO 8601, like a reasonable person.  See <https://tex.stackexchange.com/a/152394>.
\usepackage[style=iso]{datetime2}

% TODO: will this just force LaTeX to make worse choices?
% \usepackage[all]{nowidow}

% Source code listings with improved syntax highlighting
\usepackage[newfloat]{minted}
\usemintedstyle{pastie}

% Define a background color for `minted` listings.  See <http://ctan.org/pkg/minted> and
% <https://tex.stackexchange.com/q/150369>.
\usepackage{xcolor}
\definecolor{bg}{rgb}{0.95,0.95,0.95}
\setminted{bgcolor=bg}
% Don't shade the background when using `mintinline`.
\setmintedinline{bgcolor={}}

% ...
\usepackage[titletoc]{appendix}

% Keep using ISO 8601 consistently, like an even more reasonable person.  See
% <https://tex.stackexchange.com/q/231208>.
\usepackage[date=edtf, urldate=edtf, seconds=true]{biblatex}
\addbibresource{paper.bib}

\usepackage[xindy, toc, acronym]{glossaries} % Load after `hyperref`.

\makeglossaries
\loadglsentries{tex/glossary}

\usepackage{tikz}
\usetikzlibrary{datavisualization}
% \usetikzlibrary{datavisualization.formats.functions}

% See <https://tex.stackexchange.com/a/155317>, <https://tex.stackexchange.com/a/320521>
% and <https://tex.stackexchange.com/a/75507>.
% \usepackage{tikzscale}

\newcommand*{\article}{article} % report, paper?

% Define a command that takes exactly 2 arguments, the first one defaulting to `1`.  The
% second argument should be a list of items separated by `,`.  The list item at the
% position specified by the first argument in printed.  See
% <https://tex.stackexchange.com/a/99271>,
% <http://mirrors.ctan.org/macros/generic/listofitems/listofitems-en.pdf>, and
% <https://tex.stackexchange.com/q/276697>.
\usepackage{listofitems}
\newcommand*{\alts}[2][1]{%
   \setsepchar{,}%
   \readlist*\arg{#2}%
   \arg[#1]%
}

% Top matter {{{1
\title{Basics of Hardware Cache Optimization}
% \title{Algorithms for Hardware Caches}
% \title{Basics of CPU Cache Optimization}
% \title{Basics of Optimizing for Hardware Caches}
% \title{Fundamentals of Optimizing for Hardware Caches}
% \title{Introduction to Optimizations for Hardware Caches}
% \title{Optimizing for Hardware Caches}
% \subtitle{The Basics}
\author{Lukas Waymann}

% Body {{{1
\begin{document}
\maketitle

% \newpage

\begin{abstract}
   % What?
   Typical present-day CPUs have two or more levels of caches.  This \article{} presents
   the basic techniques used to optimize program performance based on knowledge about how
   these hardware caches function.

   The abstract \gls{emm} for memory hierarchies is explained and a small selection of
   algorithms \alts{developed for it, analyzed under it} are explored mathematically and
   empirically.
\end{abstract}
\newpage

\tableofcontents
\newpage

\glsresetall % Reset the use status of all acronyms.

\section{Introduction}
% What is this paper about?  Why learn about them?  How much performance is at stake?

% What are hardware caches?  Why caches?
A hardware cache is a \alts{comparatively, relatively} fast and small physical memory.  It
stores a subset of the data present in slower, larger memory that is expected to be used
again soon.  The purpose of this additional memory is to reduce the number of accesses to
the underlying slower storage.

% Hardware caches aren't going away.
There are fundamental reasons that having one single, \alts{uniform, homogeneous} type of
memory is not viable.  No signal can propagate faster than the speed of light.  Thus,
every storage technology can only reach a finite amount of data within a desired access
latency~\cite[2]{afmh}.

The most ubiquitous example for hardware caches \alts{is the hierarchy, are the various
levels (most commonly 2 or 3)} of CPU caches that are found on almost all present-day
CPUs.  They are designated L1 cache, L2 cache, and so on, with L1 being the fastest and
smallest level.  The underlying storage for CPU caches is the main memory.

There are more storage levels that \alts{comprise, constitute} the \emph{memory hierarchy}
of a computer along with CPU caches and main memory.  For example \glspl{hdd} and
\glspl{ssd}.
% Also: registers, internal buffers of HDDs and SSDs, (tapes), ...
% Focus on CPU caches.  Why?
However, swapping to \glspl{hdd} and \glspl{ssd} continues to become somewhat less common
as main memory sizes increase.  Even non-server systems can currently support 64 GiB of
main memory, eliminating the need for swapping to disk under many workloads.

I will focus on how to use CPU caches effectively and the \alts{enabled, resulting}
performance gains in this \article{}.

% TODO: what about TLB?

% vim: tw=90 sts=-1 sw=3 et fdm=marker

\section{Motivation} % So what?
% > Three things Really Matter for performance.  The first one is Algorithm, the second
% > one is your code being Non-Blocking, and the third one is Data Locality."
%      -- http://ithare.com/c-performance-common-wisdoms-and-common-wisdoms/

Hardware caches are managed by hardware directly.  They are generally opaque to the
operating system and other programs.  That is, software has no direct control over the
contents of a hardware cache.

% So what?  Why learn about hardware caches?  How much performance do I gain/lose
% depending on how cache-friendly my algorithm/code is?
\alts{Despite this, We will see that despite this}, two algorithms solving the same
problem with the same asymptotic complexity (in the same \(\Theta(g(n))\)) may differ in
performance by two orders of magnitude because of different \emph{memory access
patterns} (\cref{sec:map})~\cite{bigos}.  We will see an example of this in
\cref{sec:vvl}.

In a nutshell, hardware caches are ubiquitous but the performance improvements they
provide are conditional.
\begin{comment}
   To use them effectively,
   % To obtain optimal performance,
   algorithms must be designed and implemented with the architecture
   % design, structure, manner of functioning, inner workings, properties
   of hardware caches in mind.
\end{comment}
Effective use of hardware caches requires knowledge about \alts{how they work, their
architecture}.  Algorithms must be designed and implemented observing \alts{this
knowledge, their interactions with hardware caches}.

% How?

% vim: tw=90 sts=-1 sw=3 et fdm=marker

\section{Types of CPU Caches}
% > Most modern desktop and server CPUs have at least three independent caches: an
% > instruction cache to speed up executable instruction fetch, a data cache to speed up
% > data fetch and store, and a translation lookaside buffer (TLB) used to speed up
% > virtual-to-physical address translation for both executable instructions and data.
%      -- https://en.wikipedia.org/wiki/CPU_cache#Overview
% > There are three common types of CPU caches: ...
%      -- Scott Meyers (talk at code::dive)
Current x86 CPUs \alts{generally, typically, commonly} have three main types of caches:
data caches, instruction caches, and \glspl{tlb}%
~\cite[\href{https://youtu.be/WDIkqP4JbkE?t=11m07s}{11:07}]{scott-meyers-talk}.
Some caches are used for data as well as instructions and are called \emph{unified}.%
~\cite[20]{drepper2007}.
\alts{{A processor may have multiple caches of each type, which}, {Multiple caches of each
type may be present, and}} are organised into numerical \emph{levels}
\alts{{starting at 1, the smallest and fastest level,},}
based on their size and speed.
% Each added level is bigger and slower than its predecessor.
% The smallest and fastest is level 1.

% TODO?  The reason to have multiple levels...

% > Often there are separate L1 caches for instructions and data
%      -- Algorithms for Memory Hierarchies, page 3
% > Systems nowaeays have at-least two levels of cache
%      -- Algorithms for Memory Hierarchies, page 172
% > [T]he caches from L2 on are unified caches which contain both code and data
%      -- Drepper, p. 31
% > Later Intel models have shared L2 caches for dual-core processors.  For quad-core
% > processors we have to deal with separate L2 caches for each pair of two cores.
%      -- Drepper, p. 35

% Terminology / Nomenclature.
In practice, a \alts{currently, presently} representative%
\footnote{%
   % https://en.wikipedia.org/wiki/Bobcat_(microarchitecture)
   E.g. for AMD Family 14h processors~\cite[30--32]{14h},
   % https://en.wikipedia.org/wiki/List_of_AMD_CPU_microarchitectures
   % https://en.wikipedia.org/wiki/Zen_(microarchitecture)
   % 32 KiB L1d, 64 KiB L1i, 512 KiB L2, 8 to 16 MiB L3
   AMD Zen (17h)~\cite{zen}, and
   % https://en.wikipedia.org/wiki/Kaby_Lake
   % https://en.wikipedia.org/wiki/Skylake_(microarchitecture)
   % 32 KiB L1d, 32 KiB L1i, 256 KiB L2, 2 to 8 MiB L3
   Intel Skylake desktop processors%
   ~\cite[figure 2-1, table 2-4]{skylake}
   % ~\cite[figure 2-1, \pno~2-2, table 2-4, \pno~2-6]{skylake}.
   % ~\cite[{2-2}, {2-6}]{skylake}.
   % <https://en.wikipedia.org/wiki/Bulldozer_(microarchitecture)> is too weird.
}
x86 cache hierarchy consists of:
\begin{itemize}
   % https://en.wikipedia.org/wiki/Cache_hierarchy#Shared_versus_private
   \item Separate level 1 data and instruction caches of 32 to 64 KiB for each core
      (denoted \gls{l1d} and \gls{l1i} by  \textcite[14--15]{drepper2007}).
      % TODO?  Why have a separate instruction cache?
      Machine instructions in \gls{l1i} are already decoded%
      ~\cite[31, 56]{drepper2007}.
      % ~\cite[14, 31, 56]{drepper2007}.
   % \item A level 2 cache for \say{both code and data}~\cite[31]{drepper2007}.
   \item A unified \gls{l2} cache of 256 to 512 KiB for each core.
   \item Often a unified \gls{l3} cache of 2 to 16 MiB shared between all cores.
   \item TODO: Some \glspl{tlb} I guess.
\end{itemize}

% \subsection{Access Times}
% http://ithare.com/infographics-operation-costs-in-cpu-clock-cycles/
% http://www.getitwriteonline.com/archive/040201hyphadj.htm
\alts{Estimates, Order-of-magnitude estimates} of typical access latencies \alts[2]{are as
follows, are given by \textcite{ithare-cycles}.}%
\footnote{%
   Intel~\cite[table 2-4]{skylake},
   \textcites
   % {ithare-paadl}{ithare-wisdoms}
   [\href{https://youtu.be/WDIkqP4JbkE?t=17m52s}{17:52}, slide 18]{scott-meyers-talk}
   [2--3, 171]{afmh}[16, 20--21]{drepper2007} all give comparable numbers for various
   architectures.
   % [\ppno~16, 20--21, fig. 3.10]{drepper2007}
}

\begin{center}
   \begin{tabular}{ r | c c c c }
             & \gls{l1d} & \gls{l2} & \gls{l3} & Main Memory \\ \hline
      Cycles & 3--4      & 10--12   & 30--70   & 100--150
   \end{tabular}
\end{center}
%
% These are taken from~\textcite{ithare-cycles} but comparable numbers are given by

% > [Instruction] cache is much less problematic than the data cache.
%      -- Drepper, p. 31
The biggest target for optimizations is the data cache.  \say{[Instruction] cache is much
less problematic}~\cite[31]{drepper2007} and optimizations for data and instruction cache
tend to improve \gls{tlb} usage as well%
~\cite[\href{https://youtu.be/WDIkqP4JbkE?t=11m53s}{11:53}]{scott-meyers-talk}.

My laptop's AMD E-450 CPU has cores with \alts{an \gls{l1d} cache of 32 KiB, 32 KiB of
\gls{l1d} cache} and a unified \gls{l2} cache of 512 KiB each.%
\footnote{\Cref{app:cpuinfo} explains how to obtain this information.}
We can both verify these sizes and get \alts{a reasonably good measure, a rough measure,
an approximation} of the access times by profiling
\cref{lst:access-times}
% the \alts{program, listing} below
for different values of \mintinline{text}{SIZE}.%
% FIXME: the label is wrong: should be "Appendix" but is "Section".
\footnote{\Cref{app:cycles} details how.}
% It reads random memory addresses
% The program randomly reads
% locations
% The program repeatedly \alts{reads, accesses} random elements
This program repeatedly \alts{reads, accesses} elements
from \alts[2]{an array of the configured size, a thusly sized array}
in random \alts{order, sequence, succession}.
To do this
\alts{with minimal overhead, efficiently}, the array \alts{is first set up, acts} as a
circular, singly linked list where every element except the last \alts{points to a, has a}
random successor.  When compiled with \mintinline{text}{-DBASELINE}, only this
initialization is done.

% XXX: hacks!  Use the figure environment so that LaTeX won't display this listing after
% the plot showing the results of profiling it.  "LaTeX [only] keeps all floats of the
% same *type* in order" [1].  Use `\captionof` to label the listing correctly as a
% listing.
%
% [1]: https://tex.stackexchange.com/q/127742/#comment290982_127744
\begin{comment}
   \begin{figure}
      \inputminted[firstline=9]{c}{access-times/access-times.c}
      \captionof{listing}{TODO}
      \label{lst:access-times}
   \end{figure}
\end{comment}

% \newenvironment{code}{\captionsetup{type=listing}}{}
% \begin{code}
%    \inputminted[firstline=9]{c}{access-times/access-times.c}
%    \caption{TODO}
%    \label{lst:access-times}
% \end{code}

% XXX: hacks!  FIXME: I want to have the same amount of space after adding a caption with
% \captionof to a minted listing as there would be after a normal floating minted listing.
% See <https://tex.stackexchange.com/a/162074>.
% \captionsetup[listing]{aboveskip=5pt, belowskip=\baselineskip}

% Actually, don't use a float.  The listing should be allowed to span multiple pages which
% floats aren't.
%
% [1]: https://tex.stackexchange.com/q/14522/#comment484569_75880
% [2]: https://tex.stackexchange.com/q/175650
%      "How to allow page break inside a float environment?"
% [3]: https://tex.stackexchange.com/q/12428
\begin{center} % XXX: hack to get normal spacing after the caption and before the listing.
   \inputminted[firstline=12]{c}{access-times/access-times.c}
   \captionof{listing}{}
   % \captionof{listing}{Random Reads\label{lst:access-times}}
   \label{lst:access-times}
\end{center}

% https://tex.stackexchange.com/a/82473
% \global\csname @topnum\endcsname 0

\Cref{fig:access-times} shows the \alts{difference, deltas} of CPU cycles used when and
when not having defined \mintinline{text}{BASELINE}.  That is, only the cycles used by the
main loop are counted, not those for initialization.
I divided by \mintinline{text}{N} to get the cycles spend \alts{per, on each} loop
iteration.

Up to 32 KiB, each access takes almost exactly 3 cycles.%
\footnote{TODO: table of actual data in appendix?}
This is the \gls{l1d} access \alts{time, latency}.  At 32 KiB (the size of the \gls{l1d})
the time increases to about 3.4 cycles.  This is not surprising since the cache is shared
with other processes and the operating system, so some of our data gets evicted.  The
first dramatic increase happens at 64 KiB followed by smaller increases at 128 and 256
KiB.  I suspect we are seeing a mixture of \gls{l2} and \gls{l1d} accesses, with less and
less \gls{l1d} hits and an \gls{l2} access time of around 25 cycles.

The values from 512 KiB to 128 MiB \alts{exhibit, follow} a similar pattern.  The relative
increase when the array size matches that of the \gls{l2} is more striking than for the
\gls{l1d} before; possibly because \gls{l2} is a unified cache that also holds
instructions.  Eventually, more and more accesses go to main memory, \alts{causing,
incurring} delays of up to 200 cycles.

TODO~\cite[cf.][17]{drepper2007}.

% XXX: hacks!  Explicit placement specifier without `t` (top) to prevent the figure from
% interrupting the source code listing.
\begin{figure}[hbp]
   \centering
   \begin{tikzpicture}
      \datavisualization
      [scientific axes=clean,
       x axis={logarithmic,
               ticks={major={at={2048 as 2, 8192 as 8, 32768 as \textbf{32},
                                 131072 as 128, 524288 as \textbf{512}, 2097152 as 2048,
                                 8388608 as 8192, 33554432 as 32768, 134217728 as
                                 % $128\cdot 2^{10}$}}},
                                 % $2^{17}$}}},
                                 131072}}},
               grid={at={32768, 524288}},
               label={Array Size (KiB)}, length=0.8\textwidth},
       y axis={include value=0, label={Cycles / Iteration}, length=6cm, grid=at ticks},
       % y axis={logarithmic,
       %         ticks={major={at={3, 15, 23, 190}}},
       %         grid={at={15, 23}},
       %         label={Cycles / Element}, length=6cm},
       visualize as scatter,
       scatter={style={mark=*, mark options={scale=.65}}}]
         data [read from file=access-times/access-times.csv, separator=\space];
   \end{tikzpicture}
   % \caption{Sequential Read Access}
   % \caption{Access Times for Random Reads (\Cref{lst:access-times})}
   \caption{Access Times for Random Reads}
   \label{fig:access-times}
\end{figure}

% vim: tw=90 sts=-1 sw=3 et fdm=marker

\section{Basic Concepts}
% \section{Basic Principles}
% \section{Key Terms} % That sounds boring.
Some architectural properties of hardware caches lead to important concepts for using them
effectively.

\subsection{Cache Line} % or Cache Block
% https://en.wikipedia.org/wiki/CPU_cache#Cache_entries
%
% > On x86/x64, cache line is 64 bytes for many years now.
%      -- http://ithare.com/c-for-games-performance-allocations-and-data-locality/
% > In early caches these lines were 32 bytes long; nowadays the norm is 64 bytes.
%      -- Drepper, p. 15
% > It is not possible for a cache to hold partial cache lines.
%      -- Drepper, p. 16
% > A cache line is the "unit" of data you transfer to a cache.
%      -- http://www.cs.umd.edu/class/sum2003/cmsc311/Notes/Memory/introCache.html
\emph{Cache lines} or \emph{cache blocks} are the unit of data transfer between main
memory and cache.  They have a fixed size, which has been \say{64 bytes for many years} on
x86/x64 CPUs~\cites{ithare-paadl}[\href{https://youtu.be/WDIkqP4JbkE?t=21m41s}{21:41}]
{scott-meyers-talk}.%
\footnote{%
% > The original Pentium 4 processor also had an eight-way set associative L2 integrated
% > cache 256 KB in size, with 128-byte cache blocks.
%      -- https://en.wikipedia.org/wiki/CPU_cache#Example
% TODO: what about the block size of main memory?  Should it be the same?
Line sizes aren't \emph{necessarily} \alts{identical, homogenous} among a CPU's caches.
The Intel Pentium 4 processor had an \gls{l1d} cache with \say{64 bytes per cache
line}~\cite[p.~9]{pentium4} but an \gls{l2} cache with \say{128 bytes per cache
line}~\cite[p.~11]{pentium4}.}
\begin{comment}
   \multiplefootnoteseparator%
   % See <https://tex.stackexchange.com/a/71015>.
   \footnote{This can also be checked on the command line:
   \mintinline{bash}{cat /proc/cpuinfo | grep cache_alignment}}
\end{comment}
% > It means that as soon as you've accessed any single byte in a cache line, all the
% > other 63 bytes are already in L1 cache
%      -- http://ithare.com/c-for-games-performance-allocations-and-data-locality/
\alts{%
   This means accessing a single uncached 32-bit integer entails loading another 60
   adjacent bytes.,
   {This means when a single byte has to be loaded, another 63 adjacent bytes will be as
   well.},
   {When, for example, accessing a single byte that isn't already cached, another 63
   adjacent bytes will be loaded.}
}
% Even when compiling C++ for 64-bit systems, `int` is typically 32-bit.

My E-450 CPU is no exception and both of its data caches have 64-byte cache lines.%
\footnote{See \cref{app:cpuinfo}.}
% \begin{minted}[gobble=3]{bash}
%    $ getconf LEVEL1_DCACHE_LINESIZE; getconf LEVEL2_CACHE_LINESIZE
%    64
%    64
% \end{minted}
%stopzone
% \footnote{%
%    \mintinline{bash}!$ getconf LEVEL1_DCACHE_LINESIZE; getconf LEVEL2_CACHE_LINESIZE!\\
%    %stopzone
%    \noindent\mintinline{bash}!64!\\
%    \noindent\mintinline{bash}!64!
% }
We can verify this quite easily.  Consider \cref{lst:line-size}.  It loops over an array
with an increment given at compile time as \texttt{STEP} and measures the processor time.
\begin{center}
% \begin{listing}
   \inputminted[firstline=27]{c}{line-size/line-size.c}%
   % \caption{%
   \captionof{listing}{%
      Loop over \mintinline{text}{array} with increment \mintinline{text}{STEP}}
   \label{lst:line-size}
% \end{listing}
\end{center}
The results for different values of \texttt{STEP} are plotted in \cref{fig:line-size}.
% Starting from a step size of 16, the time roughly halves every time the step size is
% doubled.  For the first 4 step sizes however, it is almost constant.
As expected, the time roughly halves whenever the step size is doubled --- but only from a
step size of 16.  For the first 4 step sizes, it is almost constant.

% The reason why the loops take the same amount of time has to do with memory.  The
% running time of these loops is dominated by the memory accesses to the array, not by the
% integer multiplications.
%    -- https://igoro.com/archive/gallery-of-processor-cache-effects/
This is because the run times are \alts{primarily due to, dominated by} memory accesses.
Up to a step size of 8, every 64-byte line has to be loaded.  At 16, the values we modify
are 128 bytes apart,%
\footnote{16 \texttt{int64\_t} values of 8 bytes each}
so every other cache line is skipped.  At 32, three out of four cache lines are skipped,
and so on~\cite[cf.][example 2]{gallery}.

% Talk about how the term cache line is commonly conflated to mean an appropriately sized
% and *aligned* block of main memory that can be loaded into a cache line.  This is how it
% works, right?  Or can we load blocks starting at arbitrary addresses into a cache line?
% One could even imagine loading memory that isn't even contiguous.  Most sources don't
% seem to clear any of this up.  See <https://stackoverflow.com/q/3928995> ("How do cache
% lines work?")
\begin{comment}
   The term cache line is also used to refer to \alts{identically, appropriately} sized
   blocks in main memory.
   % that may be loaded into cache.
   These blocks are fixed:
   % I.e., each byte in main memory falls exactly into one cache line.
   % The boundaries

   Cache lines in main memory are fixed.
   % Data is not loaded starting from arbitrary addresses, but only from addresses that
   % are multiples of the cache line size.
   Data blocks that are loaded don't start at arbitrary addresses, but at multiples of the
   cache line size.
\end{comment}
% See <https://en.wikipedia.org/wiki/Partition_of_a_set>.  XXX: I feel like there's some
% kind of conflation of ideas going on here.
Both cache and main memory can be thought of as being partitioned%
% \footnote{%
%    In the set-theoretic sense
% }
\ (in the set-theoretic\x{al}\ sense) % https://en.wiktionary.org/wiki/set-theoretic
into \alts[3]{cache line-sized blocks, blocks of that size, cache lines}.  \alts[2]{{That
is, d}, D}ata is \alts{not, neither} \alts{read or written, loaded} starting from
arbitrary main memory addresses,
% ...nor is it loaded into arbitrarily aligned blocks in cache.
but only from addresses that are multiples of the cache line size.
% The starting address will by a multiple of the cache line size.

% TODO.  Part of the memory address partially determines which cache line is used.  The
% binary logarithm of the number of cache sets is the amount of bits required to identify
% the cache set.  The specific line used from that set depends on the replacement policy.
% A number of least significant (lowest-order, right-most) bits are ignored, since all
% addresses that only differ in those go into the same cache line.  If the line size is
% 64, log_2(64) = 6 bits are ignored.  See Drepper, page 15.

% TODO: this probably has some implications about aligning data.

% TODO: maybe talk about the 64-bit bus width and burst mode as well.  And maybe about how
% a write requires a read if we only write part of a cache line and the optimization of
% adding dummy writes (I think Andrei talked about this).
%
% [1]: https://stackoverflow.com/q/39182060 "Why isn't there a data bus which is as wide
%      as the cache line size?"

\begin{figure}
   \centering
   \documentclass[tikz, border=1pt]{standalone}

\input{tex/datavis}

% FIXME: DRY.
\pgfkeys{%
   /pgf/number format/.cd,
   1000 sep={\,},
   min exponent for 1000 sep=4,
}

\begin{document}
\tikz \datavisualization[%
   scientific axes=clean,
   x axis={logarithmic,
           ticks={major at={1, 2, 4, 8 as \textbf{8}, 16, 64, 256, 1024},
                  minor at={32, 128, 512}},
           grid={at=8},
           label={Step Size}, length=0.8\textwidth},
   y axis={logarithmic,
           % ticks={major={at={4, 16, 64, 256}}, minor={at={8, 32, 128}}},
           % ticks={major={at={6, 12, 24, 48, 92, 184}}},
           ticks={major at={5, 10, 20, 40, 80, 160, 320}},
           grid=at ticks,
           label={Processor Time (ms)}, length=6cm},
   visualize as scatter,
   scatter={style={mark=*, mark options={scale=.65}}}]
   % See <https://tex.stackexchange.com/q/198323>.
   data [read from file=line-size/line-size.csv, separator=\space];
\end{document}

% vim: ft=tex tw=90 sts=-1 sw=3 et fdm=marker

   \caption{Processor Times for Running \Cref{lst:line-size}}
   \label{fig:line-size}
\end{figure}

% vim: ft=tex tw=90 sts=-1 sw=3 et fdm=marker

\subsection{Prefetching}
\label{sec:prefetch}

Consider a simplified version of \cref{lst:access-times} that, instead of using random
accesses, simply walks over the array sequentially.  It still follows the pointers to do
this, but the array is no longer shuffled.  The results of profiling this new program
\alts{as, in the same way as, just as, like} \cref{lst:access-times} before are
\alts{plotted, shown} in \cref{fig:seq-access-times}.%
\footnote{%
   \Vref{tab:seq-access-times} shows the numerical results.%
   % Numerical results are shown in \vref{tab:seq-access-times}.%
}

% XXX: consider possible effects of [software prefetching][1].  I think GCC doesn't enable
% this type of optimization unless `-fprefetch-loop-arrays` is explicitly specified; i.e.,
% none of the `-O` levels enables it.  See [2].  XXX: WRONG:
%
%    $ gcc -O2 -Q --help=optimizers | grep prefetch
%    -fprefetch-loop-arrays                [enabled]
%
% This is also interesting:
%
%    $ diff <(gcc -Q --help=optimizers) <(gcc -O2 -Q --help=optimizers)
%
% [1]: https://en.wikipedia.org/wiki/Cache_prefetching#Compiler_directed_prefetching
% [2]: https://gcc.gnu.org/onlinedocs/gcc/Optimize-Options.html#Optimize-Options

% XXX.  This section may be misleading: it could seem like it suggests all of the measured
% speedup is a result of prefetching.  It is a huge contributor, though.  The result from
% accessing all data in a cache line should at most be that of dividing the initial main
% memory access (about 200 cycles) between 8 separate reads.  Since reads from L1d still
% take about 3 cycles:
%    (200 + 7 * 3) / 8 = 221 / 8 = 27.625
% Since the actual numbers aren't much above 6 cycles, prefetching still has a huge effect
% (6.3 / 27.625 ~= 22.8 %).
\x{Compared to the nearly 200 cycles the random accesses caused, ...}
Until the working set size \alts{matches that of, exceeds} the \gls{l1d}, the access times
are virtually unchanged at 3 cycles, but exceeding the \gls{l1d} and hitting the \gls{l2}
\alts{increases this by, adds} no more than \alts{a single, one} cycle.
More \alts{strikingly, remarkably}, \alts{exceeding} the \gls{l2}
\alts{%
   has \alts{similarly limited, comparably little} effect,
   is \alts{com:comparably, similarly} inconsequential,
}.
The access time plateaus not much above 6 cycles \x{now} --- about \alts{\SI{3}{\percent},
3\%, 3 \%} of the maximum we saw for random reads.
 % This is in large part thanks to \emph{prefetching}.
Much of this can be explained by the improved use of cache lines: the penalty of loading a
cache line is distributed among 8 accesses now.  This \alts{%
   could at \alts{best, most} get us down to,
   can not get us down to less than,
}
\SI{12.5}{\percent}.
% \alts{In large part, To a high degree, To a great extend}, the improvements are due to
% \emph{prefetching}.
The missing improvements are due to \emph{prefetching}.

Prefetching is a \x{heuristic} technique by which CPUs \alts{predict \x{certain},
recognize predictable} access patterns and \alts{%
   preemptively push cache lines up the memory hierarchy before the program needs them,
   speculatively load data before the program needs it,
}.
% > This can work well only when the memory access is predictable, though.
%      -- Drepper, p. 23
% > Currently prefetch units do not recognize non-linear access patterns.
%      -- Drepper, p. 60
\alts{%
   {This can not work unless cache line access is predictable, though, which basically
   means \x{sequential} linear},
   {For this to work, cache line access has to be predictable, which usually means
   sequential},
   % This requires,
   % This only works if,
}%
~\cite[60]{drepper2007}.%
\footnote{%
   As an example, the most complicated \emph{stride pattern} my laptop's CPU can detect is
   % https://en.wikipedia.org/wiki/Hyphen#Suspended_hyphens
   one that skips over at most 3 cache lines (for- or backwards) and may alternate strides
   (e.g.  +1, +2, +1, +2, \ldots)~\cite[278]{14h}.
}

% > The purpose of prefetching is to hide the latency of a memory access.
% ...
% > Prefetching has one big weakness: it cannot cross page boundaries.
% ...
% > [R]egardless of how good the prefetcher is at predicting the pattern, the program will
% > experience cache misses at page boundaries
%      -- Drepper, p. 60

% > Prediction or explicit prefetching might also guess where future reads will come from
% > and make requests ahead of time; if done correctly the latency is bypassed altogether.
%      -- https://en.wikipedia.org/wiki/Cache_(computing)#Latency

% > [P]refetching [can] remove some of the costs of accessing main memory since it happens
% > asynchronously with respect to the execution of the program. It can [...] make the
% > cache appear bigger than it actually is.
%      -- Drepper, p. 14
Prefetching happens asynchronously to normal program execution~\cite[14]{drepper2007}
% > [T]he processor is able to hide most of the main memory and even L2 access latency by
% > prefetching cache lines into L2 and L1d.
%      -- Drepper, p. 23
and can therefore\x{, in principle,}\ almost completely hide the main memory latency%
~\cite[23]{drepper2007}.
This is not quite what we observe in \cref{fig:seq-access-times} because the CPU
\alts{performs, has to perform} little enough work for memory bandwidth to become the
bottleneck.
% XXX: the peak transfer rate of my ThinkPad's memory (DDR3-1333, I think) is much higher
% (should be 10666.67 MB/s) than the rate observed in this test (about 2 GB/s).
%    8 B / (6.2 / 1.65 GHz) = 8 * 1.65 GB/s / 6.2 = 105.6 ~= 2.13 GB/s
% What limits it?  How to achieve the theoretical maximum?  My understanding is that
% sequential read access like this should pretty much be the most efficient use of RAM
% possible.
%
% I compiled and ran the [STREAM][1] benchmark ([FAQ][2]) by [Dr. John D. McCalpin][3]
% recommended [here][4].  It gives similarly low data rates.
%
% [1]: https://www.cs.virginia.edu/stream/
% [2]: https://www.cs.virginia.edu/stream/ref.html
% [3]: http://www.cs.virginia.edu/~mccalpin/
% [4]: http://www.admin-magazine.com/HPC/Articles/Finding-Memory-Bottlenecks-with-Stream
% [5]: https://software.intel.com/en-us/articles/optimizing-memory-bandwidth-on-stream-triad
% [6]: https://www.nersc.gov/users/computational-systems/cori/nersc-8-procurement/trinity-nersc-8-rfp/nersc-8-trinity-benchmarks/stream/
% [7]: https://en.wikipedia.org/wiki/Memory_bandwidth
%
% TODO: I also tried [bandwidth](http://zsmith.co/bandwidth.html).
Adding some expensive operations like integer divisions every loop iteration changes that
and \alts{effectively, almost completely} levels the cycles spend per iteration across all
working set sizes.%
% I tested this.  The difference between L1d and L2 virtually disappears (~0.01 cycles)
% and exceeding the L2 increases the time per element by a single cycle.
\footnote{%
   % TODO.
   See \vref{fig:seq-access-cpu-bound}.
}

% Ubiquitous.

% > Hardware based prefetching is typically accomplished by having a dedicated hardware
% > mechanism in the processor that watches the stream of instructions or data being
% > requested by the executing program, recognizes the next few elements that the program
% > might need based on this stream and prefetches into the processor's cache.
%      -- https://en.wikipedia.org/wiki/Cache_prefetching#Types_of_cache_prefetching
\alts{What I described so far is, So far I described} \emph{hardware} prefetching.  It
uses dedicated silicon to automatically detect access patterns.  There is also
\emph{software} prefetching, which is triggered by special machine instructions that may
be inserted by the compiler or manually by the programmer.  Software prefetching is
discussed in~\cite{drepper2007}.

% > The idea of [the _mm_prefetch() intrinsic] function (actually an asm instruction from
% > x86/x64 instruction set) is to inform CPU that you're about to need certain memory
% > location.
%      -- http://ithare.com/c-for-games-performance-allocations-and-data-locality/2/

% https://gcc.gnu.org/onlinedocs/gcc/Optimize-Options.html#index-fprefetch-loop-arrays

% > Most of the time, prefetch just silently works behind the scenes, and I didn't see
% > cases when messing with prefetch at application-level would be worth the trouble.  At
% > least in theory, however, such cases do exist.
%      -- http://ithare.com/c-for-games-performance-allocations-and-data-locality/2/

% > The source for the prefetch operation is usually main memory.
%      -- https://en.wikipedia.org/wiki/Cache_prefetching

\begin{figure}
   \centering
   \input{tex/graphics/seq-access-time-plot}
   \caption{Access Times for Sequential Reads}
   \label{fig:seq-access-times}
\end{figure}

% \begin{figure}
%    \centering
%    \tikz \datavisualization[array size vs cycles plot]
%       data [read from file=seq-access-times/step8/access-times.csv, separator=\space];
%    \caption{TODO}
%    \label{fig:seq8-access-times}
% \end{figure}

% [1]: http://ithare.com/c-for-games-performance-allocations-and-data-locality/
% [2]: http://ithare.com/c-for-games-performance-allocations-and-data-locality/2/
% [3]: https://en.wikipedia.org/wiki/Cache_prefetching

% vim: tw=90 sts=-1 sw=3 et fdm=marker

\input{tex/basic-concepts/assoc.tex}
\subsection{Cache Hits and Misses}
% TODO: what happens when we miss?  How bad is it?  What about \gls{smt}?
\input{tex/basic-concepts/spat_loc.tex}
\subsection{Temporal Locality}
TODO~\cite{drepper2007}.

% vim: tw=90 sts=-1 sw=3 et fdm=marker

\subsection{Memory Access Pattern}
\label{sec:map}

% vim: tw=90 sts=-1 sw=3 et fdm=marker


\section{Example: \texttt{std::vector} vs. \texttt{std::list}}
\label{sec:vvl}

% \section{Cache Replacement Policies} % Eviction Policies/Strategies/Algorithms
% https://en.wikipedia.org/wiki/Cache_replacement_policies

\section{External Memory Model}
The \gls{emm} is a widely used
% TODO: is it?
extension of the \gls{ram} model.

\begin{listing}
   \inputminted[firstline=8]{c}{array-sum/array-sum.c}
   \caption{This is C code}
   \label{lst:array-sum}
\end{listing}

% This program is likely to have excellent spatial locality: the array...

% \begin{figure}[htb]
\begin{figure}
   \centering
   \begin{tikzpicture}
      \datavisualization
      [scientific axes=clean,
       x axis={label={Array size (KiB)}, length=0.8\textwidth},
       y axis={label={Processor time (ms)}, length=6cm},
       visualize as scatter,
       scatter={style={mark=*, mark options={scale=.65}}}]
         % See <https://tex.stackexchange.com/q/198323>.
         data [read from file=array-sum/size-time.csv, separator=\space];
   \end{tikzpicture}
   \caption{This figure took way too long to create}
   \label{fig:array-sum}
\end{figure}

% See \cref{fig:array-sum}.  What's going on here?

\section{Cache-Oblivious Algorithms} % Cache-Oblivious Model

\section{Parallel Computing}

\subsection{Cache Coherence} % Coherence or coherency?
% https://en.wikipedia.org/wiki/Cache_coherence
% Dirty.  Invalid.

\subsection{False Sharing}

\clearpage
\begin{appendices} % <https://tex.stackexchange.com/q/49643>
   \crefalias{section}{appsec} % See <https://tex.stackexchange.com/q/119513>.
   \crefalias{subsection}{appsec}
   \section{Reading Information About the CPU}
\label{app:cpuinfo}
% https://stackoverflow.com/q/7281699
% https://unix.stackexchange.com/q/167038
% https://superuser.com/q/55776
There are many ways to display information about the processor(s) the operating system is
running on.  Among others, the \mintinline{text}{lscpu(1)} and
\mintinline{text}{getconf(1)} programs and the \mintinline{text}{/proc/cpuinfo}
pseudo-file on Linux.  This is \alts{how I checked, what I used to check} my CPU's cache
line sizes, for example:
\begin{minted}[gobble=3]{bash}
   $ getconf LEVEL1_DCACHE_LINESIZE; getconf LEVEL2_CACHE_LINESIZE
   64
   64
\end{minted}
%stopzone

% vim: tw=90 sts=-1 sw=3 et fdm=marker

   % Programs: `time`, bad; `\time -v`, better; oprofile(1); gprof(1); perf; Valgrind's
% cachegrind?; ...

% Performance Analysis; Run Time Analysis; Timing Methodology; Profiling Pitfalls; Run
% Time Measuring; Benchmarking Methodology; Performance Monitoring
\section{[TODO] Profiling Methodology}
\label{app:meth}

Plenty of things can go wrong when
profiling.
% measuring \alts{execution, run} times.
These are the ones I was aware of and tried to account for while doing measurements for
this \article.
\begin{itemize}
   \item \emph{CPU frequency scaling and boosting}.
      CPU frequencies are usually dynamic these days and automatically adjusted based on
      at least workload and temperature.  This should be disabled when measuring execution
      times.
      \begin{minted}[autogobble]{text}
         # TODO
      \end{minted}
      % See [5].  Not a problem when measuring cycles.
   \item \emph{Interrupts and context switches.}  The process being timed has to share
      resources with other processes and the operating system.  A context switch will not
      only increase wall time, it will also increase cycles because of TLB flushes and
      cache evictions. % See <https://en.wikipedia.org/wiki/Context_switch>.

      Processor shielding, % See [5].
      taking the minimum of several measurements, and assigning a very high priority can
      \alts{help here, mitigate these problems}.
      % This is likely better than using nice(1).
      \begin{minted}[autogobble]{text}
         # chrt -f 99 \time -v ./program
      \end{minted}
   \item \emph{CPU jumping.}
      % See <https://youtu.be/vrfYLlR8X8k?t=33m34s>.
      CPU pinning (setting processor/thread affinity) may help. % See [2] and [5].
   \item \emph{Cache warmth.}
      % See <https://youtu.be/vrfYLlR8X8k?t=41m00s>.
      When comparing different solutions on the same data set, later ones may benefit from 
      data having been loaded into cache already.  Checking whether swapping the order of
      measurements changes the results is a good idea.
\end{itemize}
These effects may \alts{add noise to, pollute} the results, render them irreproducible, or
invalidate them completely.

General mitigation/alleviation strategy: take the minimum execution time of \alts{all, a
number of} runs; all noise is additive (TODO: not the one caused by the cache being hot).
% See [7].

\subsection{Measuring CPU cycles}
\label{app:cycles}

Frequency scaling isn't a problem here (it is when comparing things to cycles that aren't
directly proportional to frequency).  I used the \mintinline{text}{ocount(1)} event
counting tool added to the OProfile (TODO: cite) project in version 0.9.9.  [TODO] This
uses \emph{hardware performance counters}.
% `perf(1)` seems to be a newer alternative to OProfile [8].

\begin{minted}[autogobble]{text}
   ophelp
   sudo chrt -f 99 ocount -e CPU_CLK_UNHALTED program
\end{minted}

TODO~\cite[78\psqq]{drepper2007}.

% [1]:  https://en.wikipedia.org/wiki/Profiling_(computer_programming)
% [2]:  https://en.wikipedia.org/wiki/Processor_affinity
% [3]:  https://en.wikipedia.org/wiki/OProfile
% [4]:  https://en.wikipedia.org/wiki/Perf_(Linux)
% [5]:  https://github.com/JuliaCI/BenchmarkTools.jl/blob/master/doc/linuxtips.md
% [6]:  http://stackoverflow.com/q/9006596 "On `/usr/bin/time` for benchmarks and `perf`"
% [7]:  https://youtu.be/vrfYLlR8X8k "Andrei Alexandrescu - Writing Fast Code"
% [8]:  http://oprofile-list.sf.narkive.com/rVbJlHdX/oprofile-vs-perf
% [9]:  http://oprofile.sourceforge.net/about/
% [10]: https://en.wikipedia.org/wiki/Hardware_performance_counter
% [11]: http://oprofile.sourceforge.net/doc/index.html

% vim: tw=90 sts=-1 sw=3 et fdm=marker

   % \section{Disabling the Prefetcher}
My initial results from from running \cref{lst:array-sum} were quite different.  See TODO.

TODO: I installed \texttt{msr-tools} and OProfile.
% This seems to work.  It did affect performance.  See
% <https://community.amd.com/thread/180725>.
\begin{minted}[gobble=3]{bash}
   $ sudo modprobe msr
   $ sudo rdmsr -a 0xC0011022
   600001800000000
   600001800000000
   $ sudo wrmsr -a 0xC0011022 0x600001800002000
   $ sudo rdmsr -a -f '13:13' 0xC0011022
   1
   1
   $ sudo rdmsr -a 0xC0011022
   600001800002000
   600001800002000
   $ sudo ocount -e PREFETCH_INSTRUCTIONS_DISPATCHED ./a.out
\end{minted}
%stopzone
I think resetting the MSR to enable hardware prefetches doesn't work.  To get the same
benchmmark results as before tweaking it, I have to reboot.

% vim: tw=90 sts=-1 sw=3 et fdm=marker

\end{appendices}

\clearpage

\printglossary[type=\acronymtype] % Print the list of acronyms.

\printbibliography[heading=bibintoc]

\end{document}

% vim: tw=90 sts=-1 sw=3 et fdm=marker
