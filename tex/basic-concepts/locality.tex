\subsection{Locality of Reference}
%           Principle of Locality

% TODO: What's *data* locality?  Just another term for spatial locality?

% > [C]aches have proven themselves in many areas of computing because access patterns in
% > typical computer applications exhibit the locality of reference.
%      -- https://en.wikipedia.org/wiki/Cache_(computing)

% > Locality is [...] one type of predictable behavior that occurs in computer systems.
% > Systems that exhibit strong locality of reference are great candidates for performance
% > optimization through the use of techniques such as the caching, prefetching for memory
% > and advanced branch predictors at the pipelining stage of processor core.
%      -- https://en.wikipedia.org/wiki/Locality_of_reference

% > Realizing that locality exists is key to the concept of CPU caches as we use them
% > today. -- Drepper, p. 14

% > The memory hierarchy wouldn't be very effective if two facts about programs weren't
% > true.  Programs exhibit spatial locality and temporal locality.
%      -- http://www.cs.umd.edu/class/sum2003/cmsc311/Notes/Memory/introCache.html

% > Programs which have good spatial locality benefit from the fact that data are
% > transferred from main memory to cache in blocks, while programs which have good
% > temporal locality benefit from the fact that caches hold several blocks of data.
%      -- Algorithms for Memory Hierarchies, page 171

\alts{%
  {Two properties exhibited by \x{the memory access patterns of} computer code to varying
  degrees \alts{
    distinctly impact,
    are particularly \alts{essential, crucial} for,
    merit emphasis because of their impact on,
  } cache effectiveness:
  these are \emph{spatial locality} and \emph{temporal locality}.},
  {Spatial locality is a property \x{exhibited by} \alts{computer code exhibits, programs
  exhibit} to varying degrees.},
}
\alts{%
  {Both are measures of how well the code's memory access pattern matches certain
  principles.},
  {Both are measures of how well certain \alts{principles, ideals} about memory access are
  matched.\x{  For spatial locality, these are:}},
}

% This one goes first because it's really more fundamental than spatial locality.
\subsubsection{Temporal Locality}

% Temporal locality refers to the reuse of specific data, and/or resources, within a
% relatively small time duration.
%      -- https://en.wikipedia.org/wiki/Locality_of_reference

% > A sequence of references exhibits temporal locality of recently accessed data are
% > likely to be accessed again in the near future.
%      -- Algorithms for Memory Hierarchies, page 215

% > Even if the memory used over short time periods is not close together there is a high
% > chance that the same data will be reused before long (temporal locality).
%      -- Drepper, p. 14

% One access to a memory location \alts{suggests, deserves} another within a short time
% frame.
One access suggests another.  That is, \alts{once, previously} \alts{referenced, accessed}
memory locations tend to be used again within a short time frame.
%
This is \alts{really, effectively, in fact} the \alts{intrinsic, fundamental}
\alts[2]{raison d'être of memory hierarchies, motivation for having a memory hierarchy in
the first place}.  When \alts{a cache line, some data} is loaded \x{into cache} but not
accessed again before being evicted, the cache \alts{provided no benefit, was useless}.
% If every memory location were only used once

\subsubsection{Spatial Locality}

% > [S]patial locality refers to requests for data physically stored close to data that
% > has been already requested.
%      -- https://en.wikipedia.org/wiki/Cache_(computing)

% > Spatial locality refers to the use of data elements within relatively close storage
% > locations.  Sequential locality, a special case of spatial locality, occurs when data
% > elements are arranged and accessed linearly, such as, traversing the elements in a
% > one-dimensional array.
%      -- https://en.wikipedia.org/wiki/Locality_of_reference

% On spatial locality:
%
% > When a data item is accessed, it is likely that data items in sequential memory
% > locations will also be accessed.
%      -- https://en.wikibooks.org/wiki/Microprocessor_Design/Cache

% > Data accesses are also ideally limited to small regions.
%      -- Drepper, p. 14

% > Spatial locality: When a block is accessed, it should contain as much useful data as
% > possible.
%      -- Algorithms for Memory Hierarchies, page 9

% > A sequence of references exposes spatial locality if data located close together in
% > address space tend to be referenced close together in time.
%      -- Algorithms for Memory Hierarchies, page 215

% \alts{Spatial locality, It} is a measure of how
% \alts{%
%   % predictable the memory access \alts{is, patterns are}.
%   well two \alts{principles, assumptions} about memory access are matched:
% }
\begin{enumerate*}[font=\bfseries]
  \item For each accessed memory location, nearby locations are used as well within a
    short time frame.
  % \item Memory locations that are accessed are close to recently accessed ones.
  \item Memory is accessed sequentially.
\end{enumerate*}
% [S]patial locality is one of the principles on which caches are based.
%      -- Drepper, p. 15
We have already seen in the last two sections that caches \alts{%
  take advantage of \alts{both these principles, spatial locality} by design,
  are designed to take advantage of \alts{these principles, spatial locality},
% }~\cite[15]{drepper2007}:
}:
\begin{enumerate}[font=\bfseries]
  \item Data is loaded in blocks; subsequent accesses to locations in an already loaded
    cache line are basically free.
  \item \alts{Cache lines, Locations} from sequential access patterns are prefetched
    \alts{ahead of time, speculatively}.
\end{enumerate}

\subsubsection{Notes}
% \subsubsection{Instruction Cache}

% Talk about how instructions naturally have good spatial locality and are therefore less
% problematic and less of a target for optimizations from an application programmer's
% point of view.

% > Code always has quite good spatial and temporal locality.
%      -- Drepper, p. 31

% > [I]nstructions usually run sequentially (with the occasionaly branch or jump).  Since
% > instructions are contiguous in memory, they exhibit spatial locality.  When you run
% > one instruction, you're likely to run the next one too.  That's the nature of running
% > programs.
%      -- http://www.cs.umd.edu/class/sum2003/cmsc311/Notes/Memory/introCache.html

% On temporal locality:
%
% > For code this means, for instance, that in a loop a function call is made and that
% > function is located elsewhere in the address space.  The function may be distant in
% > memory, but calls to that function will be close in time.
%      -- Drepper, p. 14

% As said before, instruction cache \alts{tends to benefit less from, is less dependent
% on} optimization.

\alts{%
  Access to instructions,
  {\alts{Access to \x{the}, The} machine code itself, which is also cached, },
}
inherently has good spatial locality~\cite[31]{drepper2007} since \alts{they are,
instruction are, it is}
\alts{%
  \x{naturally} executed sequentially outside of jumps,
  mostly executed sequentially (with some jumps),
},
and good temporal locality\x{~\cite[31]{drepper2007}}\ because of loops and function
calls~\cite[14]{drepper2007}.

% [1]: https://en.wikipedia.org/w/index.php?title=Cache_memory&oldid=778902517#Functional_principles_of_the_cache_memory

% Data is loaded from bigger, slower memory into smaller, faster memory (e.g. from main
% memory into the CPU cache) in \emph{blocks}.

% Moving down the memory hierarchy, access latencies increase faster than the
% \emph{obtainable} bandwidth: \say{[w]e can still achieve large bandwidths by accessing
% many close-by bits together [...].  Access to large \emph{blocks} [emphasis added] of
% memory is almost as fast as access to a single bit}.~\cite[2]{afmh}.

% Consider the program shown in \cref{lst:array-sum}.  It repeatedly loops over an array
% to compute the sum of its elements.  Before exiting, it prints the CPU time spent
% summing the array.

% vim: tw=90 sts=-1 sw=3 et fdm=marker
