\subsection{Prefetching}
\label{sec:prefetch}

Consider a simplified version of \cref{lst:access-times} that, instead of using random
accesses, simply walks over the array sequentially.  It still follows the pointers to do
this, but the array is no longer shuffled.  The results of profiling this new program
\alts{as, in the same way as, just as, like} \cref{lst:access-times} before are
\alts{plotted, shown} in \cref{fig:seq-access-times}.%
\footnote{%
   \Vref{tab:seq-access-times} shows the numerical results.%
   % Numerical results are shown in \vref{tab:seq-access-times}.%
}

% XXX: consider possible effects of [software prefetching][1].  I think GCC doesn't enable
% this type of optimization unless `-fprefetch-loop-arrays` is explicitly specified; i.e.,
% none of the `-O` levels enables it.  See [2].  XXX: WRONG:
%
%    $ gcc -O2 -Q --help=optimizers | grep prefetch
%    -fprefetch-loop-arrays                [enabled]
%
% This is also interesting:
%
%    $ diff <(gcc -Q --help=optimizers) <(gcc -O2 -Q --help=optimizers)
%
% [1]: https://en.wikipedia.org/wiki/Cache_prefetching#Compiler_directed_prefetching
% [2]: https://gcc.gnu.org/onlinedocs/gcc/Optimize-Options.html#Optimize-Options

% XXX.  This section may be misleading: it could seem like it suggests all of the measured
% speedup is a result of prefetching.  It is a huge contributor, though.  The result from
% accessing all data in a cache line should at most be that of dividing the initial main
% memory access (about 200 cycles) between 8 separate reads.  Since reads from L1d still
% take about 3 cycles:
%    (200 + 7 * 3) / 8 = 221 / 8 = 27.625
% Since the actual numbers aren't much above 6 cycles, prefetching still has a huge effect
% (6.3 / 27.625 ~= 22.8 %).
\x{Compared to the nearly 200 cycles the random accesses caused, ...}
Until the working set size \alts{matches that of, exceeds} the \gls{l1d}, the access times
are virtually unchanged at 3 cycles, but exceeding the \gls{l1d} and hitting the \gls{l2}
\alts{increases this by, adds} no more than \alts{a single, one} cycle.
More \alts{strikingly, remarkably}, \alts{exceeding} the \gls{l2}
\alts{%
   has \alts{similarly limited, comparably little} effect,
   is \alts{com:comparably, similarly} inconsequential,
}.
The access time plateaus not much above 6 cycles \x{now} --- about \alts{\SI{3}{\percent},
3\%, 3 \%} of the maximum we saw for random reads.
 % This is in large part thanks to \emph{prefetching}.
Much of this can be explained by the improved use of cache lines: the penalty of loading a
cache line is distributed among 8 accesses now.  This \alts{%
   could at \alts{best, most} get us down to,
   can not get us down to less than,
}
\SI{12.5}{\percent}.
% \alts{In large part, To a high degree, To a great extend}, the improvements are due to
% \emph{prefetching}.
The missing improvements are due to \emph{prefetching}.

Prefetching is a \x{heuristic} technique by which CPUs \alts{predict \x{certain},
recognize predictable} access patterns and \alts{%
   preemptively push cache lines up the memory hierarchy before the program needs them,
   speculatively load data before the program needs it,
}.
% > This can work well only when the memory access is predictable, though.
%      -- Drepper, p. 23
% > Currently prefetch units do not recognize non-linear access patterns.
%      -- Drepper, p. 60
\alts{%
   {This can not work unless cache line access is predictable, though, which basically
   means \x{sequential} linear},
   {For this to work, cache line access has to be predictable, which usually means
   sequential},
   % This requires,
   % This only works if,
}%
~\cite[60]{drepper2007}.%
\footnote{%
   As an example, the most complicated \emph{stride pattern} my laptop's CPU can detect is
   % https://en.wikipedia.org/wiki/Hyphen#Suspended_hyphens
   one that skips over at most 3 cache lines (for- or backwards) and may alternate strides
   (e.g.  +1, +2, +1, +2, \ldots)~\cite[278]{14h}.
}

% > The purpose of prefetching is to hide the latency of a memory access.
% ...
% > Prefetching has one big weakness: it cannot cross page boundaries.
% ...
% > [R]egardless of how good the prefetcher is at predicting the pattern, the program will
% > experience cache misses at page boundaries
%      -- Drepper, p. 60

% > Prediction or explicit prefetching might also guess where future reads will come from
% > and make requests ahead of time; if done correctly the latency is bypassed altogether.
%      -- https://en.wikipedia.org/wiki/Cache_(computing)#Latency

% > [P]refetching [can] remove some of the costs of accessing main memory since it happens
% > asynchronously with respect to the execution of the program. It can [...] make the
% > cache appear bigger than it actually is.
%      -- Drepper, p. 14
Prefetching happens asynchronously to normal program execution~\cite[14]{drepper2007}
% > [T]he processor is able to hide most of the main memory and even L2 access latency by
% > prefetching cache lines into L2 and L1d.
%      -- Drepper, p. 23
and can therefore\x{, in principle,}\ almost completely hide the main memory latency%
~\cite[23]{drepper2007}.
This is not quite what we observe in \cref{fig:seq-access-times} because the CPU
\alts{performs, has to perform} little enough work for memory bandwidth to become the
bottleneck.
% XXX: the peak transfer rate of my ThinkPad's memory (DDR3-1333, I think) is much higher
% (should be 10666.67 MB/s) than the rate observed in this test (about 2 GB/s).
%    8 B / (6.2 / 1.65 GHz) = 8 * 1.65 GB/s / 6.2 = 105.6 ~= 2.13 GB/s
% What limits it?  How to achieve the theoretical maximum?  My understanding is that
% sequential read access like this should pretty much be the most efficient use of RAM
% possible.
%
% I compiled and ran the [STREAM][1] benchmark ([FAQ][2]) by [Dr. John D. McCalpin][3]
% recommended [here][4].  It gives similarly low data rates.
%
% [1]: https://www.cs.virginia.edu/stream/
% [2]: https://www.cs.virginia.edu/stream/ref.html
% [3]: http://www.cs.virginia.edu/~mccalpin/
% [4]: http://www.admin-magazine.com/HPC/Articles/Finding-Memory-Bottlenecks-with-Stream
% [5]: https://software.intel.com/en-us/articles/optimizing-memory-bandwidth-on-stream-triad
% [6]: https://www.nersc.gov/users/computational-systems/cori/nersc-8-procurement/trinity-nersc-8-rfp/nersc-8-trinity-benchmarks/stream/
% [7]: https://en.wikipedia.org/wiki/Memory_bandwidth
%
% TODO: I also tried [bandwidth](http://zsmith.co/bandwidth.html).
Adding some expensive operations like integer divisions every loop iteration changes that
and \alts{effectively, almost completely} levels the cycles spend per iteration across all
working set sizes.%
% I tested this.  The difference between L1d and L2 virtually disappears (~0.01 cycles)
% and exceeding the L2 increases the time per element by a single cycle.
\footnote{%
   % TODO.
   See \vref{fig:seq-access-cpu-bound}.
}

% Ubiquitous.

% > Hardware based prefetching is typically accomplished by having a dedicated hardware
% > mechanism in the processor that watches the stream of instructions or data being
% > requested by the executing program, recognizes the next few elements that the program
% > might need based on this stream and prefetches into the processor's cache.
%      -- https://en.wikipedia.org/wiki/Cache_prefetching#Types_of_cache_prefetching
\alts{What I described so far is, So far I described} \emph{hardware} prefetching.  It
uses dedicated silicon to automatically detect access patterns.  There is also
\emph{software} prefetching, which is triggered by special machine instructions that may
be inserted by the compiler or manually by the programmer.  Software prefetching is
discussed in~\cite{drepper2007}.

% > The idea of [the _mm_prefetch() intrinsic] function (actually an asm instruction from
% > x86/x64 instruction set) is to inform CPU that you're about to need certain memory
% > location.
%      -- http://ithare.com/c-for-games-performance-allocations-and-data-locality/2/

% https://gcc.gnu.org/onlinedocs/gcc/Optimize-Options.html#index-fprefetch-loop-arrays

% > Most of the time, prefetch just silently works behind the scenes, and I didn't see
% > cases when messing with prefetch at application-level would be worth the trouble.  At
% > least in theory, however, such cases do exist.
%      -- http://ithare.com/c-for-games-performance-allocations-and-data-locality/2/

% > The source for the prefetch operation is usually main memory.
%      -- https://en.wikipedia.org/wiki/Cache_prefetching

\begin{figure}
   \centering
   \input{tex/graphics/seq-access-time-plot}
   \caption{Access Times for Sequential Reads}
   \label{fig:seq-access-times}
\end{figure}

% \begin{figure}
%    \centering
%    \tikz \datavisualization[array size vs cycles plot]
%       data [read from file=seq-access-times/step8/access-times.csv, separator=\space];
%    \caption{TODO}
%    \label{fig:seq8-access-times}
% \end{figure}

% [1]: http://ithare.com/c-for-games-performance-allocations-and-data-locality/
% [2]: http://ithare.com/c-for-games-performance-allocations-and-data-locality/2/
% [3]: https://en.wikipedia.org/wiki/Cache_prefetching

% vim: tw=90 sts=-1 sw=3 et fdm=marker
