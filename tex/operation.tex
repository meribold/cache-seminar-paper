\section{Cache Operation Overview}
%        Cache Operation Synopsis
%        Cache Operation Outline
%        Cache Operation Basics
%        Operation
%        High-level Cache Operation
%        Basics of Cache Operation
%        Basic Cache Operation

% > When the [..] CPU [...] needs to access data presumed to exist in the backing store,
% > it first checks the cache.
%      -- https://en.wikipedia.org/wiki/Cache_(computing)#Operation
%
% > When the processor needs to read from or write to a location in main memory, it first
% > checks whether a copy of that data is in the cache.
%      -- https://en.wikipedia.org/wiki/CPU_cache

% > When the processor needs to read or write a location in main memory, it first checks
% > for a corresponding entry in the cache.  The cache checks for the contents of the
% > requested memory location in any cache lines that might contain that address.
%      -- https://en.wikipedia.org/wiki/CPU_cache

% > If the CPU needs a data word the caches are searched first.
%      -- Drepper, section 3.2: "Cache Operation at High Level", page 15
%
% > A cache hit occurs when the requested data can be found in a cache, while a cache miss
% > occurs when it cannot.
%      -- https://en.wikipedia.org/wiki/Cache_(computing)

Whenever
\alts{%
  a program \alts{requests, accesses} a memory address,
  {\alts{a program requires, accessing} data at some address,},
}
the CPU will search its caches. % It has to do a search, in general.
If the \alts{location is present\x{in cache}, location is cached, data is present}, a
\emph{cache hit} occurs.  Otherwise, the result is a \emph{cache miss} and
% If unsuccessful, a \emph{cache miss} occurs and
the next level of the memory hierarchy, which could be another CPU cache, is tried.
% we fall back to the next level of the memory hierarchy (which could be another CPU cache).

% > By default all data read or written by the CPU cores is stored in the cache.
%      -- Drepper, section 3.2: "Cache Operation at High Level", page 15
Unless \alts{explicitly, specifically, deliberately} \alts{prevented, disabled},
% TODO: how is it prevented?
% \alts{{ and}, {,}}
% \alts{%
%   with some exceptions only \alts{relevant to, concerning} OS programming,
%   with some exceptions in OS programming,
%   with no exceptions outside of OS programming,
%   as far as application programmer's are concerned,
%   in the context of non-systems programming,
%   ignoring OS programming,
%   disregarding OS programming,
% },
\alts{%
  the CPU brings all accessed data into cache,
  the CPU will cache all accessed data,
  all accessed data will be loaded into cached,
  all memory accessed is cached,
}
(with some exceptions only relevant to OS programming)%
~\cite[15]{drepper2007}.
% > To be able to load new data in a cache it is almost always first necessary to make
% > room in the cache.
%      -- Drepper, section 3.2: "Cache Operation at High Level", page 16
This
happens in response to cache misses and
will\x{almost always},
\alts{much more often than not, in the vast majority of cases}, cause another cache entry
to be \emph{evicted} and replaced~\cite[16]{drepper2007}.

% TODO.  Writes are more complicated than this, right?  There's some kind of buffering, I
% think.

% TODO: talk about cache replacement strategies?  See
% <https://en.wikipedia.org/wiki/Cache_replacement_policies>

% TODO: this section is very short.

% vim: tw=90 sts=-1 sw=3 et fdm=marker
